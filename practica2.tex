%! Author = esteve
%! Date = 23/10/22

% Preamble
\documentclass[12pt]{article}

% Packages
\usepackage[utf8]{inputenc}
\usepackage[a4paper, margin=0.75in]{geometry}

\title{Práctica 2}
\author{Esteve Soria Fabián}
\counterwithin*{section}{part}
\begin{document}
    \maketitle


    \section{Breve descripción explicando los contenidos del documento}
    \section{Justificación del conjunto de atributos final elegido y su rango utilizado para la definici´on de los
    estados.}

    En el desarrollo del proyecto he probado diferentes atributos para tener una
    Q table pequeña pero que pudiera representar la mayor información útil posible.

    En la primera iteración utilicé unicamente la dirección de los fantasmas.
    Esta resulta útil para casi todos los mapas ya que el agente acaba siendo capaz
    de alcanzar al fantasma menos cuando dicho fantasma se encuntra detrás de un muro.

    La siguiente iteración es una mejor sobre la anterior donde en vez de usar las
    4 direcciones en las cuales los fantasmas se encuentran se usan 8: norte, noreste,
    este, sureste, etc.
    Así se permite más granularidad en las observación de los fantasmas por
    parte del agente.

    En la siguiente iteración del los atributos usados para la Q table se añade
    la acción del fantasma más cercano.
    La hipótesis por la cual se añade este valor es porque permite al agente
    \"predecir\" donde el fantasma se dirige y así esperar que pueda aprender a
    adelantarse al movimiento del fanstasma.




    \section{Descripción de la función de refuerzo final empleada.}


    \section{Descripción del código desarrollado.}
    \section{Descripción de los resultados (puntuación obtenida por pacman en los mapas proporcionados, comentarios
    sobre el comportamiento de pacman)}


    \section{Conclusiones.}
\end{document}