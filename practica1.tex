%! Author = esteve
%! Date = 23/10/22

% Preamble
\documentclass[12pt]{article}

% Packages
\usepackage[utf8]{inputenc}
\usepackage[a4paper, margin=0.75in]{geometry}

\title{Práctica 1}
\author{Esteve Soria Fabián}
\counterwithin*{section}{part}
\begin{document}
    \maketitle

    \section*{Ejercicio 1}
    \subsection{¿Cuántas celdas/estados aparecen en el tablero?
    ¿Cuántas acciones puede ejecutar el agente?
    Si quisieras resolver el juego mediante aprendizaje por refuerzo, ¿cómo lo harías?}

    Aparecen 4x3 celdas, por tanto un total de 12 celdas o estados. \\
    El agente puede ejecutar 4 posibles acciones: mover arriba, derecha, abajo e izquierda.

    Dado que se trata de un entorno con
    
    \subsection{Abrir el fichero qlearningAgents.py y buscar la clase QLearningAgent. Describir los métodos que
    aparecen en ella.}
    
    \textbf{readQtable:} Este método lee la Qtable de la memoria no volátil del sistema que ha sido guardada por
    writeQtable. \\
    \textbf{writeQtable:} Escribe la Qtable a la memoria no volátil del sistema que se genera y(o actualiza en el
    método update).  \\
    \textbf{computePosition:} Este método devuelve la linea de la Qtable a la que corresponde una posición (o estado).\\
    \textbf{getQValue:} Devuelve el Q value para un par estado acción almacenado en la Q table. \\
    \textbf{computeValueFromQValues:} Devuelve la mayor Q para un estado determinador de la Q table. \\
    \textbf{computeActionFromQValues:} Devuelve la mejor acción posible para un estado determinado basado en el Q
    value. \\
    \textbf{getAction:} Devuelve la mejor acción posible o una acción random dependiendo del valor epsilon. \\
    \textbf{update:} Método que se llama para actualizar la Qtable después de una transición entre estados. \\
    \textbf{getPolicy:}  Lo mismo que computeActionFromQValues. \\
    \textbf{getValue:} Lo mismo que computeValueFromQValues. \\

    \subsection{Ejecuta ahora el agente anterior con:}
    \begin{verbatim}
    python gridworld.py -a q -k 100 -n 0
    \end{verbatim}
    \subsection{¿Qué información se muestra en el laberinto? ¿Qué aparece por terminal cuando se realizan los
    movimientos en el laberinto?}
    En cada celda aparecen 4 valores que indican el Q value de cada acción para dicho estado. En el caso de los dos
    estados finales solo hay un valor que tenderá a la recompensa.

    En el terminal aparece el estado, la acción que se toma, el estado en el que se acaba y la recompensa para dicha
    transición.
    \subsection{¿Qué clase de movimiento realiza el agente anterior?}
    El agente sigue un movimiento aleatorio.
    \subsection{¿Se pueden sacar varias políticas óptimas? Describe todas las políticas óptimas para este problema.}
    En este problema existen varias políticas óptimas ya que el único refuerzo positivo se consigue en el estado
    final y salvo una celda que tiene refuerzo negativo todas las demás tienen un refuerzo neutro.
    Cualquier política que no lleve al agente al recuadro con recompensa negativa y su nodo final sea el de
    recompensa será óptima.
    \subsection{Escribir el metodo update de la clase QLearningAgent utilizando las funciones de actualización del algoritmo
    Q-Learning. Para ello, inserta el código necesario allí donde aparezca la etiqueta INSERTA TU CODIGO
    AQUÍ siguiendo las instrucciones que se proporcionan, con el fin de conseguir el comportamiento deseado.}
    \subsection{Establece en el constructor de la clase QLearningAgent el valor de la variable epsilon a 0,05.
    Ejecuta nuevamente con:}
    \begin{verbatim}
    python gridworld.py -a q -k 100 -n 0
    \end{verbatim}
    \subsection*{¿Qué sucede?}
    El valor epsilon, cuanto más alto, más movimiento aleatorio hace el agente en vez de la acción más indicada por
    la política de ese momento. Dado que el valor es bajo el agente tiene a seguir el camino de acciones más óptimo
    conocido y tiende a explorar menos.
    Esto hace que el agente tenga un sesgo por el camino que ya conoce y le cueste más encontrar el otro camino óptimo.
    \subsection{Después de la ejecución anterior, abrir el fichero qtable.txt. ¿Qué contiene?}
    Contiene la Q table de dicha ejecución donde hay 12 filas, una fila por cada estado. Además, tiene 5 columnas,
    una por cada acción y la última indica el valor de reward de los valores terminales.

    \section*{Ejercicio 2}
    En el ejercicio anterior, siempre que el agente decidía moverse hacia una dirección se movía en esa dirección
    con probabilidad 1. Es decir, se trataba de un MDP determinista. Ahora vamos a crear un MDP estocástico:
    \subsection{Ejecuta y juega un par de partidas con el agente manual:}
    \begin{verbatim}
    python gridworld.py -m -n 0.3
    \end{verbatim}
    \subsection*{¿Qué sucede? ¿Crees que el agente QLearningAgent seré capaz de aprender en este nuevo escenario?}
    El agente no se mueve donde se le dice que se mueva, existe una posibilidad de que dicha acción no ocurra y otra
    aleatoria ocurra.
    Será capaz de aprender pero más lentamente ya que aunque quisiera hacer la política óptima a veces haría un paso
    en la dirección incorrecta y no recibiría la recompensa que debería por hacer dicha acción.
    \subsection{Reiniciar los valores de la tabla Q del fichero qtable.txt. Para ello ejecutar desde el terminal:}
    \begin{verbatim}
    cp qtable.ini.txt qtable.txt
    \end{verbatim}
    \subsection{Ejecutar el agente QLearningAgent:}
    \begin{verbatim}
    python gridworld.py -a q -k 100 -n 0.3
    \end{verbatim}
    \subsection{Tras unas cuantos episodios, ¿se genera la política óptima? Y si se genera, ¿se tarda más o menos
    que en el caso determinista?}
    Se acaba generando la política óptima pero tarda mucho más por lo indicado en el punto anterior. Al tener un
    comportamiento estocástico el agente no recibe las recompensas adecuadas para el par acción estado usado y eso
    relentiza la convergencia.

\end{document}